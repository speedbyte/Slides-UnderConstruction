\begin{enumerate}
\item ~
  \section{Chapter 12}\label{chapter-12}

  \begin{enumerate}
  \item ~
    \subsection{Validation}\label{validation}
  \end{enumerate}
\end{enumerate}

Overview 1

\begin{enumerate}
\def\labelenumi{\arabic{enumi}.}
\item
  Building a Convincing Safety Case 3
\item
  Formal Methods 7
\item
  Testing 11
\item
  Fault Injection 15
\item
  Dependability Analysis 20
\end{enumerate}

Points to Remember 23

\begin{enumerate}
\item ~
  \subsection{}\label{section}

  \subsection{Overview}\label{overview}
\end{enumerate}

\begin{itemize}
\item
  Is this system fit for its purpose?
\item
  System must be proven to be trustworthy
\item
  Convincing safety case
\item
  Formal methods for ultra-dependable real-time system
\item
  Testing real-time systems, test data selection
\item
  Dependability analysis, hazards and risks, FTA and FMEA
\end{itemize}

\textbf{12.1} \protect\hypertarget{teil2}{}{}\textbf{Building a
Convincing Safety Case}

A safety case is a combination of a sound set of arguments concerning
the safety of a given design.

The safety case must convince an independent certification authority
that the system is safe to deploy.

\textbf{Outline of the Safety Case}

Computer systems can fail for external or internal reasons.

External reasons are related to the operational environment (stress,
EMI, temperature) and to the system specification.

Internal reasons are

\begin{enumerate}
\def\labelenumi{\arabic{enumi}.}
\item
  Computer hardware failure due to random physical fault. Remedy as part
  of the safety case: redundancy.
\item
  Hardware or software contains residual design faults. This is the real
  challenge. No single validation technology can provide the required
  evidence that a computer system will meet the ultra-high dependability
  requirement (see chapter 1).
\end{enumerate}

The safety case must thus combine evidence from independent sources to
convince the certification authorities that the system is safe to
deploy. These sources are:

\begin{enumerate}
\def\labelenumi{\arabic{enumi}.}
\item
  \begin{quote}
  A disciplined software development process with inspections and design
  reviews
  \end{quote}
\end{enumerate}

\textbf{\\
}

\begin{enumerate}
\def\labelenumi{\arabic{enumi}.}
\item
  \begin{quote}
  Experimental evidence from testing combined with structural arguments
  about the partitioning of the system in autonomous error-containment
  regions.
  \end{quote}
\item
  \begin{quote}
  Formal analysis of critical properties
  \end{quote}
\item
  \begin{quote}
  Experienced dependability of previous generations of similar systems.
  \end{quote}
\item
  \begin{quote}
  Experimental data about field-failure rates of critical components as
  input for reliability models that demonstrate that the system will
  mask random component failures with the required high probability.
  \end{quote}
\item
  \begin{quote}
  Hardware and design diversity to reduce probability of common-mode
  design failures
  \end{quote}
\end{enumerate}

\textbf{Properties of the Architecture}

No single fault must cause a catastrophic failure. For fail-safe
applications, safe state must be entered before the consequences of
error affect system behavior. For fail-operational applications, safe
system service must be provided even after a single fault.

\textbf{Error Containment Regions:} At architectural level, it must be
demonstrated that every single fault can only affect a defined error
containment region, and will be detected at the boundaries of this error
containment region.

Typical weak points that can lead to common-mode failure of all nodes
within a distributed system:

\begin{itemize}
\item
  A single source of time, such as a central clock.
\item
  A babbling node.
\item
  A single fault in the power supply or the grounding system.
\item
  A single design error that is replicated when the same hardware or
  system software is used in all nodes.
\end{itemize}

ask with no synchronisation point within, e.g. no semaphore ``wait''
operation, mutexes or other constructs that could make the task wait or
block. Execution time can be determined in isolation, i.e. without
knowledge of behaviour of other tasks.

\textbf{Example:} Four nodes connected by a replicated bus. If
communication controller implements an event-triggered protocol, a
single faulty host can disrupt communication between all nodes by
sending high-priority messages at arbitrary points in time.

If the ARINC 629 protocol is implemented, errors can still occur if the
communication controller accesses the bus randomly. In the TTP a bus
guardian would prevent such an error.

\includegraphics[width=7.18750in,height=3.20833in]{media/image1.png}

\textbf{Composability:} Helps in designing a convincing safety case.
Assume the nodes can be grouped into nodes implementing safety-critical
functions, and into nodes that do not contain safety-critical functions.

If it can be shown, at the architectural level, that no error in any one
of nodes not containing safety-critical functions can affect the proper
operation of any node containing safety-critical functions, than it is
possible to exclude the nodes that do not contain such functions from
further considerations during the safety case analysis.

\textbf{12.2} \protect\hypertarget{teil3}{}{}\textbf{Formal Methods }

Formal methods: the use of mathematical and logical techniques to
express, investigate, and analyze the specification, design,
documentation, and behaviour of computer hardware and software.

\textbf{Formal Methods in the Real World}

Formal investigation of real-world phenomena requires the following
steps:

\begin{enumerate}
\def\labelenumi{\arabic{enumi}.}
\item
  Conceptual model building: Informal first step, see chapter 4, leads
  to reduced natural language representation of the real-world phenomena
  under investigation. Note: all assumptions, omissions, or
  misconceptions that are introduced here will remain in the model, and
  limit the validity of the conclusions derived from the model
  (assumption coverage).
\item
  Model formalization: Natural language representation is transformed
  into a formal specification language with precise syntax and
  semantics.
\item
  Analysis of the formal model: Problem is formally analyzed. In
  computer systems, the analysis methods are based on discrete
  mathematics and logic. In other disciplines, analysis methods are
  based on different branches of mathematics, e.g., differential
  equations to analyse control problems.
\item
  Interpretation of the results.
\end{enumerate}

Only step 3 can be mechanized. All other steps involve human involvement
and human intuition, and are thus fallible.

Ideal and complete verification environment would look like this:

\begin{itemize}
\item
  Express specification in formally defined specification language.
\item
  Write implementation in formally defined implementation language.
\item
  Establish mechanically consistency between specification and
  implementation (e.g., MDA).
\item
  Ensure that all assumptions of the target machine are consistent with
  the model of computation as defined by the implementation language.
\item
  Ensure that the verification environment itself is correct.
\end{itemize}

\textbf{Classification of Formal Methods}

Formal methods can be classified into three levels:

\begin{enumerate}
\def\labelenumi{\arabic{enumi}.}
\item
  \begin{quote}
  Use of concepts and notation from discrete mathematics. Ambiguous
  natural language statements about requirements and specification of a
  system are replaced by the symbols and conventions from discrete
  mathematics and logic, e.g. set theory, relations, and functions.
  \end{quote}
\item
  \begin{quote}
  Use of formalized specification languages with some mechanical support
  tools. A formal specification language with a fixed syntax is used,
  that allows the mechanical analysis of some properties of the problems
  expressed in the specification language. However, it is not possible
  to mechanically generate complete proofs.
  \end{quote}
\item
  \begin{quote}
  Use of fully formalized specification languages with comprehensive
  support environments, including mechanized theorem proving or proof
  checking. A precisely defined specification language with a direct
  interpretation in logic is supplied, and a set of support tools is
  provided to allow the mechanical analysis of specifications expressed
  in the formal language.
  \end{quote}
\end{enumerate}

\textbf{Benefits from the Application of Formal Methods}

\textbf{Level (i) Methods:}

\begin{itemize}
\item
  Use of level(i) methods improves communication within the project team
  and within an engineering organization, an enriches quality of
  documentation.
\item
  Particularly in the early stage of a project the need for precise and
  effective communication is great, when the mechanical control system
  and the computer system are specified.
\item
  The concepts of discrete mathematics (set, relation, etc.) are precise
  and yet abstract. This avoids misunderstandings and ambiguities.
\item
  Some simple mechanical analysis is possible (undefined symbols,
  uninitialized variables), and can lead to the detection of
  inconsistencies and omissions.
\item
  Reviews are more effective if requirements are expressed in a precise
  notation rather than ambiguous natural language.
\item
  It is more difficult to express vague ideas and immature concepts in a
  semiformal notation.
\end{itemize}

\textbf{Level (ii) Methods:} a task with no synchronisation point
within, e.g. no semaphore ``wait'' operation, mutexes or other
constructs that could make the task wait or block. Execution time can be
determined in isolation, i.e. without knowledge of behaviour of other
tasks.

\textbf{Level (iii) Methods:} a task with no synchronisation point
within, e.g. no semaphore ``wait'' operation, mutexes or other
constructs that could make the task wait or block. Execution time can be
determined in isolation, i.e. without knowledge of behaviour of other
tasks.

\textbf{12.3} \protect\hypertarget{teil4}{}{}\textbf{Testing }

System architecture design is primarily interface design.

An interface between two subsystems of a real-time system is
characterized by

\begin{itemize}
\item
  The control properties (control signals crossing interface, what do
  they do)
\item
  The temporal properties (temporal constraints to be satisfied by
  control signals and data crossing the interface)
\item
  The functional intent (intended functions of the interface partner)
\item
  The data properties (structure and semantics of data elements crossing
  the interface)
\end{itemize}

In case events can occur any time, we speak of a dense time base.

In case events are permitted to occur only in certain time intervals π,
we speak of a sparse time base.

In a dense time base it can be difficult to establish temporal order.

Example: The functional intent of a node in an engine controller is to
guarantee the car conforms to environmental standards. As the standards
change, e.g. new laws are passed, the function of the node may have to
change, but not its functional intent.

The functional intent is thus at a higher level of abstraction than the
function.

The functional intent can be related to a ``goal'' in requirements
engineering.

\textbf{The Probe Effect}

In many cases, the interfacing partners use different syntactic
structures and incompatible coding schemes to represent information that
must cross the interface.

An intelligent interface component must be placed between the interface
partners to transform the different representations. This we call a
resource controller.

\includegraphics[width=6.87222in,height=3.21389in]{media/image2.png}

Resource controllers act as gateways between two different subsystems
with different representations.

\textbf{Design for Testability}

We distinguish between real world interfaces, and message interfaces.

Example: specific man-machine interface (SMMI) would be a concrete world
interface, with touchpad, screen, buttons. A generalized man-machine
interface (GMMI or abstract message interface) would just consider the
messages that cross the interface, and their temporal properties.

\textbf{Test Data Selection}

To improve compatibility between systems designed by different
manufacturers, some international standard organizations have attempted
to standardize message interfaces. In automotive systems, the interfaces
are typically bus interfaces.

\textbf{Peak Load:} a task with no synchronisation point within, e.g. no
semaphore ``wait'' operation, mutexes or other constructs that could
make the task wait or block. Execution time can be determined in
isolation, i.e. without knowledge of behaviour of other tasks.

\textbf{Worst Case Execution Time (WCET):} a task with no
synchronisation point within, e.g. no semaphore ``wait'' operation,
mutexes or other constructs that could make the task wait or block.
Execution time can be determined in isolation, i.e. without knowledge of
behaviour of other tasks.

\textbf{Fault-Tolerant Mechanisms:} a task with no synchronisation point
within, e.g. no semaphore ``wait'' operation, mutexes or other
constructs that could make the task wait or block. Execution time can be
determined in isolation, i.e. without knowledge of behaviour of other
tasks.

\textbf{What can be Inferred from ``Perfect Working''}

To improve compatibility between systems designed by different
manufacturers, some international standard organizations have attempted
to standardize message interfaces. In automotive systems, the interfaces
are typically bus interfaces.

\textbf{12.4} \protect\hypertarget{teil5}{}{}\textbf{Fault Injection}

Example rolling mill with three drive controllers and associated
controller nodes and pressure sensors:

\textbf{Why Fault Injection?}

To improve compatibility between systems designed by different
manufacturers, some international standard organizations have attempted
to standardize message interfaces. In automotive systems, the interfaces
are typically bus interfaces.

\textbf{Automotive transport protocols (OSI level 4)}

\begin{longtable}[c]{@{}llll@{}}
\toprule
\textbf{Transport Protocol} & \textbf{Application} & \textbf{European
Standards} & \textbf{US Standards}\tabularnewline
ISO TP & CAN busses & ISO 15765-2 &\tabularnewline
SAE J1939 & CAN busses & & SAE J1939/21\tabularnewline
TP 1.6

TP 2.0 & CAN busses & Manufacturer standard

VW/Audi/Seat/Skoda

Base is OSEK COM 1.0 &\tabularnewline
\bottomrule
\end{longtable}

\textbf{Physical Fault Injection}

A system where all control signals are derived from event triggers is
called an event-triggered (ET) system. Event examples: button pushed,
activation of limit switch, arrival of new message at a node, completion
of task within node.

\textbf{Injected Faults:} a task with no synchronisation point within,
e.g. no semaphore ``wait'' operation, mutexes or other constructs that
could make the task wait or block. Execution time can be determined in
isolation, i.e. without knowledge of behaviour of other tasks.

\textbf{Example:} a task with no synchronisation point within, e.g. no
semaphore ``wait'' operation, mutexes or other constructs that could
make the task wait or block. Execution time can be determined in
isolation, i.e. without knowledge of behaviour of other tasks.

A system where all control signals are derived from time triggers is
called a time-triggered (TT) system. A time trigger is a control signal
derived from the progression of time, e.g. the clock within a node
reaches a preset point in time.

Example: computer system controlling an elevator.

\includegraphics[width=7.26944in,height=3.11736in]{media/image3.png}

\textbf{The Hardware under Test:} a task with no synchronisation point
within, e.g. no semaphore ``wait'' operation, mutexes or other
constructs that could make the task wait or block. Execution time can be
determined in isolation, i.e. without knowledge of behaviour of other
tasks.

\textbf{Experimental Setup:} a task with no synchronisation point
within, e.g. no semaphore ``wait'' operation, mutexes or

\includegraphics[width=7.98819in,height=3.39583in]{media/image4.png}

\textbf{Results:} a task with no synchronisation point within, e.g. no
semaphore ``wait'' operation, mutexes or

\textbf{Software Implemented Fault Injection}

In an ET system, the external event triggers are often relayed to the
computer system via interrupts. An interrupt is an asynchronous
hardware-supported request for a specific task activation caused by an
event external (i.e. outside the node) to the currently active
computation.

Interrupts cause an overhead (worst case administrative overhead, WCAO),
mostly due to the required context switches.

If interrupt frequency is too high, no CPU capacity remains for actual
computations. Interrupts are outside the sphere of control of the node,
which makes it difficult to handle such overload conditions.

\textbf{12.5} \protect\hypertarget{teil6}{}{}\textbf{Dependability
Analysis}

Deadlines can only be guaranteed if worst case execution (WCET) times of
all application tasks are known a priori.

In addition, the worst case delays caused by administrative functions
(e.g. operating system services, context switches, scheduling) need to
be known. These delays we call the worst case administrative overhead
(WCAO).

\textbf{Fault Tree Analysis}

WCET depends on

\begin{itemize}
\item
  source code of task
\item
  properties of object code generated by compiler
\item
  characteristics of compiler
\end{itemize}

\includegraphics[width=7.27639in,height=3.46319in]{media/image5.png}

\textbf{Failure Mode and Effect Analysis (FMEA)}

WCET depends on

\begin{itemize}
\item
  source code of task
\item
  properties of object code generated by compiler
\item
  characteristics of compiler
\end{itemize}

\begin{longtable}[c]{@{}lllll@{}}
\toprule
\textbf{Component} & \textbf{Failure Mode} & \textbf{Failure Effect} &
\textbf{Probability} & \textbf{Criticality}\tabularnewline
& & & &\tabularnewline
& & & &\tabularnewline
& & & &\tabularnewline
\bottomrule
\end{longtable}

\protect\hypertarget{teil9}{}{}\textbf{Points to Remember}
