% Pr�sentation-Vorlage Universit�t T�bingen
% erstellt von Ekaterina Kohler (ekaterina.kohler@googlemail.com)
\documentclass[20pt,t]{beamer}
\usepackage{praesentation}
%-------------!!!!BITTE AUSF�LLEN!!!!------------------
% hier f�ngt die Eingabe der Grundinformationen zur Pr�sentation an
\newcommand{\fakult}{<Name der Fakult�t, Dezernat oder SFB>}	%Fakult�t
\newcommand{\fbname}{<Fachbereich/Institut/Lehrstuhl>}	%Fachbereich oder Name des Instituts
\newcommand{\fussnote}{Autor/Verfasser/Thema/Rubrik/Titel etc.} %Text der Fusszeile auf den Innenfolien
\newcommand{\crtext}{\copyright\ 2012 Universit�t T�bingen } %Copyright-Text
\newcommand{\kooplogoUnten}{\colorbox{cyan}{\parbox{0.2\textwidth}{\centering \fontsize{12pt}{12pt}\selectfont MAX. ZWEI externe Koop.-Logos}}} %Kooperations-Logo in der Fusszeile auf der Titelseite. Statt \colorbox MAX. ZWEI externe Koop.-Logos. Falls kein Koop.-Logo verwendet werden soll, lassen Sie die zweite geschweifte Klammer leer

\begin{document}
\begin{frame} %Titelseite
\author{Verfasser [optional]} %Autor/Verfasser
\title{Headline (max. zweizeilig/linksb�ndig)} %Titel
\subtitle{Untertitel maximal zwei Zeilen} %Untertitel
\date{23.07.2012} %Datum
\newcommand{\titelbild}{{\color{grau}\rule{\textwidth}{70mm}}\\[-0.4cm]{\color{rot}\rule{\textwidth}{5mm}}} %statt grauer Box ein Deckblattbild einf�gen und die rote Linie nach oben versetzen, hier um 0.35cm
\maketitle \\
\vspace*{35mm} %Abstand zwischen dem Untertitel und Datum/Verfasser. Hier 35mm
{\fontsize{16}{16}\selectfont\bf\insertdate , \insertauthor} \hfill \kooplogoUnten
\end{frame}

\begin{frame}{Testing} %zweite Folie
False Positives
  \begin{itemize}
  \item[$\bullet$] The false positives are results of test cases which are returned positive although it should have been negative
    \begin{itemize}
    \item[-] Anstrich Ebene 2
     \begin{itemize}
     	\item[\tiny$\blacksquare$] {Anstrich Ebene 3}
     \end{itemize}
    \end{itemize}
  \item[$\bullet$] Text
  \end{itemize}
Text linksb�ndig
\end{frame}

\begin{frame}{Headline (max. zweizeilig/linksb�ndig)} %dritte Folie
\blindtext
\end{frame}
\begin{frame}[c] %R�ckfolie
\colorbox{rot}{\parbox{\textwidth}{\centering\vspace*{1.5cm} \parbox{12cm}{\color{white} {\huge Danke.} \bigskip \\ Kontakt: \bigskip \\ {\bf \fakult}\\ \fbname\\ Musterstra�e 00, 72074 T�bingen \\ Telefon: +49 7071 29-0000\\ Telefax: +49 7071 29-00\\
beispiel@uni-tuebingen.de}\vspace*{1.5cm}}}
\end{frame}

\begin{frame}{Testing} %zweite Folie
False Positives and False Negatives
  \begin{itemize}
  \item[$\bullet$] The false positives are results of test cases that are returned positive although they should have been reported negative.
  \item[$\bullet$] The false negatives are results of test cases that are returned negative although they should have been reported positive. 
  \end{itemize}
Text linksb�ndig
\end{frame}


%------------------------------------------------------
\end{document}
