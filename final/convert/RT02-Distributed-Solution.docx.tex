\begin{enumerate}
\item ~
  \section{Chapter 2}\label{chapter-2}

  \begin{enumerate}
  \item ~
    \subsection{Distributed Real-Time
    Systems}\label{distributed-real-time-systems}
  \end{enumerate}
\end{enumerate}

Overview 2

\begin{enumerate}
\def\labelenumi{\arabic{enumi}.}
\item
  System Architecture 3
\item
  Composability 9
\item
  Scalability 12
\item
  Dependability 16
\item
  Physical Installation 19
\end{enumerate}

Points to Remember 21

\begin{enumerate}
\item ~
  \subsection{}\label{section}

  \subsection{Overview}\label{overview}
\end{enumerate}

\begin{itemize}
\item
  From functional point of view there is no difference between
  centralized architecture and distributed architecture
\item
  Overview of distributed real-time system architecture with a set of
  nodes and communication network interconnecting these nodes
\item
  Introduction to concept of event messages and state messages
\item
  Composable architecture as preferred alternative for hard real-time
  systems
\item
  The communication network interface between host computer in node and
  communication network
\item
  Implications on scalability
\item
  Implications on dependability
\end{itemize}

\textbf{2.1} \protect\hypertarget{teil2}{}{}\textbf{System Architecture}

In a distributed system, it is feasible to encapsulate the logical
function and associated computer hardware into s single unit, a node.

All essential functional and temporal properties are hidden behind a
simple and stable node interface.

The nodes in a distributed architecture are less complex than they would
be in a centralized architecture. Functional failure can be better
pinpointed to failure of a single node.

A distributed application can be decomposed into an operator cluster, a
computational cluster, and a controlled object cluster. In case the
computational cluster is implemented as a distributed system, the system
looks like shown below.

\includegraphics[width=6.26736in,height=3.00069in]{media/image1.png}

\protect\hypertarget{ux5fux5fFieldmarkux5fux5f8ux5f1847268712}{}{}\includegraphics[width=10.01875in,height=5.89375in]{media/image2.png}

\textbf{Hardware structure}

A node can be portioned into two subsystems:

\begin{itemize}
\item
  host computer
\item
  local communication controller
\end{itemize}

\includegraphics[width=4.50694in,height=2.65556in]{media/image3.png}

The communication-network interface (CNI) is located at the transport
level of the OSI reference model.

The set of all communication controllers, along with the physical
interconnections, form the real-time communication system of the
cluster.

\textbf{Communication-network interface}

Purpose of the real-time communication system: Transport messages from
the CNI of the sender node to the CNI of the receiver node

\begin{itemize}
\item
  within predictable time
\item
  with a small latency jitter
\item
  with high reliability
\end{itemize}

\textbf{Data semantics}

Two types of messages at the CNI: event messages and state messages.

Every event is significant; event messages must not be lost (loss of
synchronization).

Event messages must be queued at receiver and be removed upon reading.

Order in queue must correspond to temporal order of event occurrence.

State messages (e.g. current temperature) may overwrite old state
messages. State values may be read more than once.

In real-time systems, state semantics is needed more frequently than
event semantics.

\textbf{Control strategy}

When must a message be sent? Decision either within sphere of control of
host computer (external control) or within sphere of control of
communication system (autonomous control).

Most common used strategy is external control.

With external control, a ``send'' command issued by the host computer to
the CNI initiates transmission of message. When a message arrives at
receiver, a control signal (interrupt) from the communication system
crosses the CNI and unblocks a ``receive'' command.

With autonomous control, the communication system decides autonomously
when to send the next message, and when to deliver the message at the
CNI of the receiver. Autonomous control is normally time-triggered. The
communication system contains a transmission schedule.

\textbf{Event message}

Event message combines event semantics with external control. Event
messages require one-to-one synchronization between sender and receiver.

\textbf{State message}

State message combines state-value semantics with autonomous control.
State messages can be considered similar to global variables, except for

\begin{itemize}
\item
  The communication system guarantees atomicity of a message write
  operation
\item
  There is only a single sender (writer) process
\end{itemize}

There is no need for one-to-one synchronization; loose coupling between
nodes.

\textbf{Communication system}

A number of design alternatives are available:

\begin{itemize}
\item
  single channel system, such as bus or ring
\item
  multiple channel system, such as a mesh network
\end{itemize}

Increase communication reliability by message retransmission in case of
failure, or by always replicating messages, or replicating message
channels.

Communication system is a critical resource of a distributed system.
Reliability should be an order of magnitude higher than the reliability
of individual nodes.

\textbf{Gateways}

A gateway exchange relative views between two interacting clusters. A
gateway node has either

\begin{itemize}
\item
  an instrumentation interface and a communication interface (interface
  node)
\end{itemize}

or

\begin{itemize}
\item
  two communication interfaces, each interfacing to one of the
  interacting clusters
\end{itemize}

The gateway host transforms data formats of one cluster to those
expected by the other cluster. Gateways are often used to hide internal
features of legacy systems.

\textbf{2.2} \protect\hypertarget{teil3}{}{}\textbf{Composability }

In an architecture supporting composability it is ensured that
properties of a subsystem are maintained during system integration.
Thus, large systems can be built by composition of a number of smaller,
well tested subsystems.

\includegraphics[width=6.77222in,height=2.78264in]{media/image4.png}

An architecture is said to be composable with respect to a specified
property if the system integration will not invalidate this property
once the property has been established at the subsystem level (e.g.
timeliness or testability).

In a composable architecture, system properties follow from subsystem
properties.

\textbf{Event triggered communication systems}

In an event-triggered (ET) communication system temporal control is
external to the communication system.

Access conflicts can occur when several host computers decide to send a
message simultaneously. The probability of access conflicts increases
with the number of nodes and messages on the network.

Temporal control in an ET system is a global issue, depending on the
behaviour of all nodes in the distributed system. From the point of view
of temporal control ET systems are thus not composable. Example: CAN
bus.

\textbf{Time-triggered communication systems}

In a time-triggered communication system, temporal control resides
within the communications system, and is independent from the
application software of the host computers.

State messages are transported at predetermined points in time from the
sender CNI to the receiver CNI.

The CNI acts thus as a temporal firewall, isolating the temporal
behaviour of the host computer from the temporal behaviour of the
network.

System integration will not change the temporal properties of the CNI. A
TT architecture is thus composable with respect to communication
timeliness.

\textbf{2.3} \protect\hypertarget{teil4}{}{}\textbf{Scalability}

Most automotive systems evolve over an extended period of time (years or
decades). Evolving requirements are thus the rule. Functions are being
modified, and new functions are being added. A scalable architecture is
open to such changes, and does not impose an upper limit on the
capabilities of the system.

\includegraphics[width=5.64167in,height=1.76528in]{media/image5.png}

\textbf{Extensability}

Means:

\begin{itemize}
\item
  nodes can be added within the given capacity of the communication
  channel, to add processing power to the system
\item
  new clusters can be added using gateway nodes when a cluster has
  reached it communication capacity limit, or the processing power limit
\end{itemize}

without changing the original nodes and networks.

\textbf{Complexity}

Complexity of a system relates to the number of parts, and the number of
types of interactions among the parts.

In a scalable architecture, the effort to understand a single function
remains constant and does not grow with the size of the system.

Complexity can be reduced by encapsulating functionality behind simple
and stable interfaces.

The partitioning of a system into subsystems, the encapsulation of the
subsystems, the preservation of the abstractions in case of faults, and
strict control over the interaction patterns among the subsystems are
the key mechanism for controlling complexity.

Thus, time-triggered architectures can handle complexity better than
event-triggered architectures.

\textbf{Silicon cost}

The die size of an integrated circuit determines its cost. The cost of a
die is approximately:

Cost = K · (die area)\textsuperscript{3}

The die area required for the communication controller is increasingly
unimportant with regard to total cost. The cost of a distributed
architecture starts out higher, but increases slower with system size
than that of a centralized architecture.

However, there are many tradeoffs determining the system cost, and
silicon cost is becoming an increasingly less important factor here.

\includegraphics[width=4.77500in,height=2.45278in]{media/image6.png}
\includegraphics[width=4.14306in,height=2.40764in]{media/image7.png}

Example: requirements for body computer microcontrollers:

\protect\hypertarget{ux5fux5fFieldmarkux5fux5f9ux5f1847268712}{}{}\includegraphics[width=9.52986in,height=5.38611in]{media/image8.png}

\textbf{2.4} \protect\hypertarget{teil5}{}{}\textbf{Dependability}

Implementing a dependable system requires effective fault containment,
error containment, and fault tolerance.

\textbf{Error-containment region}

Fault-tolerant system must be structured into partitions that act as
error-containment regions.

This way, errors in one partition can be detected and corrected or
masked before consequences corrupt the rest of system.

Error containment region must be specified as a service, delivered
across a small interface to the outside world. Error of the service must
be detectable at the interface.

Error-containment coverage: probability that an error in the containment
region is detected at the interface.

Error-containment regions can be introduced at level of functional
hardware block, level of task. In a distributed system, a node can
constitute an error-containment region.

\textbf{Replication}

Node failures can be masked by providing actively replicated nodes. The
replicas must show deterministic behaviour. It requires careful
hardware/software design to obtain replica determinism, i.e. replicated
nodes ``visit'' the same states at about the same time. This precludes
asynchronous preemptive scheduling.

\textbf{Certification support}

Often design of safety-critical systems must be approved by independent
certification agency.

Certification agency bases its assessment of the system on the analysis
of the safety case presented by the designer.

A safety case is the accumulation of credible analytic and experimental
evidence that convinces the certification agency that the system is safe
to deploy (see ISO 26262).

Faults are categorized (here from the aircraft industry):

\begin{itemize}
\item
  Catastrophic: fault can be cause of an accident
\item
  Hazardous: fault reduces safety margin of the redundant system,
  operation is critical
\item
  Major: fault reduces safety margin such that immediate maintenance is
  required
\item
  Minor: fault has only a small effect on the safety margin. Repair at
  next scheduled maintenance
\item
  No Effect: fault has no effect on the safety margin
\end{itemize}

Key concern is remainder of safety margin after occurrence of a primary
fault.

\includegraphics[width=2.54306in,height=2.07500in]{media/image9.png}

The consequence of a fault is an error.

Errors in non-safety critical subsystems must be detected before they
can propagate into a safety critical subsystem.

Architecture must rule out by design unintended interactions among
subsystems of different criticality.

In a centralized architecture, it is very difficult to demonstrate
error-containment.

\textbf{2.5} \protect\hypertarget{teil6}{}{}\textbf{Physical
Installation}

There are a number of reasons for physical integration of a small
microcontroller into a mechanical subsystem for compact mechatronical
components:

\begin{itemize}
\item
  Less cabling: Electromechanical parts can interface directly with
  microcontroller, without error-prone and expensive cabling and
  connections
\item
  Components become self-contained and achieve higher degree of
  autonomy. Functions can be performed and tested without the need to
  interconnect the component to a distant control system.
\item
  The communication to and from this component can be serialized and
  accomplished by a single wire or twisted-pair field bus, simplifying
  installation of the component.
\end{itemize}

Example: gear control box from Bosch/ZF

\protect\hypertarget{ux5fux5fFieldmarkux5fux5f10ux5f1847268712}{}{}\includegraphics[width=9.19375in,height=4.92431in]{media/image10.png}

\protect\hypertarget{teil8}{}{}\textbf{Points to Remember}

How can you make the various functionality composable?

What is the CNI?

If you want to make the system fault tolerant?

Fault hypothesis required?

Error containment -- how?

Temporal Requirement -- MSC ( message sequence chart )

Functional Requirements?

State messages or Event messages?

Include a new node in the system -- myEcu

All button press should be routed through my ecu

And the temporal requirement should not be affected.
