\section{Introduction: Chapter 1: Distributed Real-Time Systems}


\section{Literature for this lecture}\label{literature-for-this-lecture}

\begin{frame}
    \frametitle{Bibliography}
\tiny
	Text books for this lecture\\
\vspace*{1ex}

  \begin{thebibliography}{9}

  \beamertemplatearticlebibitems

    \bibitem{beameruserguide}
    Kopetz, H: Distributed Real-Time Systems, Springer, $2008$
    \bibitem{beameruserguide}
    Buttazo, G: Hard Real-Time Computing Systems, Springer $2005$
  \end{thebibliography}

\vspace*{2ex}
Complementing texts
Some books complementing the material treated in this lecture

\vspace*{1ex}

  \begin{thebibliography}{9}
  \beamertemplatearticlebibitems

    \bibitem{beameruserguide}
Liu, J.S.:	Real-Time Systems, Prentice Hall $2000$
    \bibitem{beameruserguide}
Verissimo, P; Rodrigues, L.: 	Distributed Systems for System Architects,
Kluwer $2001$
    \bibitem{beameruserguide}
Laplante, P.:	Real-Time Systems Design and Analysis, IEEE Press, $2004$
    \bibitem{beameruserguide}
Halbwachs, N.:	Synchronous Programming of Reactive Systems, Kluwer $1993$
    \bibitem{beameruserguide}
Zimmermann, W.; Schmidgall, R.:	Bussysteme in der Fahrzeugtechnik, Vieweg
$2006$ (German only)

  \end{thebibliography}

\vspace*{2ex}

Journal Articles and Web Documents
Original journal articles and documents from the web pertaining to this lecture

\vspace*{1ex}

  \begin{thebibliography}{9}
  \beamertemplatearticlebibitems

    \bibitem{beameruserguide}
    Albert, A.: Comparison of Event-Triggered and Time-Triggered Concepts with
    Regard to Distributed Control Systems, Embedded World, $2004$, Nuernberg,
%    \url{http://www.semiconductors.bosch.de/pdf/embedded_world_$04$_albert.pdf}
    \bibitem{beameruserguide}
    Mueller, B.; Fuehrer, T.; Hartwich, F.; Huegel, R.; Weiler, H.:    Fault
    Tolerant TTCAN Networks, Proceedings $8$th International CAN Conference; $2002$; Las
    Vegas, NV %http://www.semiconductors.bosch.de/pdf/Fault_Tolerant_TTCAN.pdf



  \end{thebibliography}

\end{frame}

\begin{frame}{Introduction}{Part 1}
    \begin{block}{Overview}
Chapter 1	  The Real-Time Environment\\
Chapter 2	  Distributed Real-Time Systems\\
Chapter 3	  Global Time\\
Chapter 4	  Modeling Real-Time Systems\\
Chapter 5	  Real-Time Entities and Images\\
Chapter 6	  Fault Tolerance\\
Chapter 7	  Real-Time Communication\\
Chapter 8	  Time-Triggered Protocols\\
Chapter 9	  Input and Output\\
Chapter 10	Real-Time Operating Systems: OSEK and AUTOSAR\\
Chapter 11	Real-Time Scheduling\\
Chapter 12	Validation
\end{block}
\end{frame}


\begin{frame}{Introduction}{Part 1}
    \begin{block}{The Real-Time Environment}
\begin{itemize}
\item
  Definition of a real-time system.
\item
  Simple model with operator, computer system, and controlled object.
\item
  Introduction of distributed real-time systems.
\item
  Hard real-time systems and soft real-time systems.
\item
  Functional, temporal, and dependability requirements.
\item
  Sphere of control
\item
  Event-triggered versus time-triggered systems.
\end{itemize}
\end{block}
\end{frame}

\begin{frame}{Introduction}{Part 2}
    \begin{block}{Distributed Real-Time Systems}
\begin{itemize}
\item
  Distributed system architecture overview, clusters, nodes,
  communication network
\item
  Structure of node with host computer, communication network interface,
  communication controller
\item
  Event and state messages, gateways.
\item
  Concept of composability.
\item
  Event- and time-triggered communication systems.
\item
  Scalability, dependability, issues of physical installation.
\end{itemize}
\end{block}
\end{frame}


\begin{frame}{Introduction}{Part 3}
    \begin{block}{Global Time}
\begin{itemize}
\item
  Notions of causal order, temporal order, and delivery order
\item
  External observers, reference clocks, and global time base
\item
  Sparse time base to view event order in a distributed real-time system
\item
  Internal clock synchronization to compensate for drift offset.
  Influence of the communication system jitter on the precision of the
  global time base.
\item
  External time synchronization, time gateways, and the Internet network
  time protocol (NTP)
\end{itemize}
\end{block}
\end{frame}


\begin{frame}{Introduction}{Part 4}
    \begin{block}{Modeling Real-Time Systems}
\begin{itemize}
\item
  Introduction of a conceptual model for real-time systems
\item
  Tasks, nodes, fault-tolerant units, clusters
\item
  Simple and complex tasks
\item
  Interface placement and interface layout
\item
  Temporal control and logical control
\item
  The history state
\end{itemize}
\end{block}
\end{frame}


\begin{frame}{Introduction}{Part 5}
    \begin{block}{Real-Time Entities and Images}
\begin{itemize}
\item
  Real-time entities
\item
  Observations, state and event observations
\item
  Real-time images as current picture of real-time entity, and real-time
  objects
\item
  Temporal accuracy and state estimation to improve real-time image
  accuracy
\item
  Permanence in case of race conditions and idempotency with replicated
  messages
\item
  Replica determinism to implement fault-tolerance by active redundancy
\end{itemize}
\end{block}
\end{frame}


\begin{frame}{Introduction}{Part 6}
    \begin{block}{Fault Tolerance}
\begin{itemize}
\item
  Failures, Errors, and Faults
\item
  Error Detection
\item
  A Node as a Unit of Failure
\item
  Fault Tolerant Units
\item
  Reintegration of a Repaired Node
\item
  Design Diversity
\end{itemize}
\end{block}
\end{frame}


\begin{frame}{Introduction}{Part 7}
    \begin{block}{Real-Time Communication}
\begin{itemize}
\item
  Real-Time Communication Requirements
\item
  Flow Control
\item
  OSI Protocols for Real-Time
\item
  Fundamental Conflicts in Protocol Design
\item
  Media-Access Protocols
\item
  Performance Comparison: ET versus TT
\item
  The Physical Layer
\end{itemize}
\end{block}
\end{frame}


\begin{frame}{Introduction}{Part 8}
    \begin{block}{Time-Triggered Protocols}
\begin{itemize}
\item
  Introduction to Time-Triggered Protocols
\item
  Overview of the TTP/C Protocol Layers
\item
  The Basic CNI
\item
  Internal Operation of TTP/C
\item
  TTP/A for Field Bus Applications
\end{itemize}
\end{block}
\end{frame}


\begin{frame}{Introduction}{Part 9}
    \begin{block}{Input and Output}
\begin{itemize}
\item
  The dual role of time
\item
  Agreement protocol
\item
  Sampling and polling
\item
  Interrupts
\item
  Sensors and actuators
\item
  Physical installation
\end{itemize}
\end{block}
\end{frame}


\begin{frame}{Introduction}{Part 10}
    \begin{block}{Real-Time Operating Systems: OSEK and AUTOSAR}
\begin{itemize}
\item
  Task management
\item
  Interprocess communication
\item
  Time management
\item
  Error detection
\item
  OSEK and AUTOSAR
\end{itemize}
\end{block}
\end{frame}


\begin{frame}{Introduction}{Part 11}
    \begin{block}{Real-Time Scheduling}
\begin{itemize}
\item
  The scheduling problem
\item
  The adversary problem
\item
  Dynamic scheduling, dynamic priority servers
\item
  Static scheduling, fixed priority servers
\end{itemize}
\end{block}
\end{frame}


\begin{frame}{Introduction}{Part 12}
    \begin{block}{Validation}
\begin{itemize}
\item
  Building a Convincing Safety Case
\item
  Formal Methods
\item
  Testing
\item
  Fault Injection
\item
  Dependability Analysis
\end{itemize}
\end{block}
\end{frame}
