\begin{enumerate}
\item ~
  \section{Chapter 4}\label{chapter-4}

  \begin{enumerate}
  \item ~
    \subsection{Modeling Real-Time
    Systems}\label{modeling-real-time-systems}
  \end{enumerate}
\end{enumerate}

Overview 2

\begin{enumerate}
\def\labelenumi{\arabic{enumi}.}
\item
  Appropriate Abstractions 3
\item
  The Structural Elements 6
\item
  Interfaces 9
\item
  Temporal Control versus Logical Control 15
\item
  Worst Case Execution Time 20
\item
  The History State 24
\end{enumerate}

Points to Remember 26

\begin{enumerate}
\item ~
  \subsection{}\label{section}

  \subsection{Overview}\label{overview}
\end{enumerate}

\begin{itemize}
\item
  Introduction of a conceptual model for real-time systems
\item
  Tasks, nodes, fault-tolerant units, clusters
\item
  Simple and complex tasks
\item
  Interface placement and interface layout
\item
  Temporal control and logical control
\item
  The history state
\end{itemize}

\textbf{4.1} \protect\hypertarget{teil2}{}{}\textbf{Appropriate
Abstractions}

A model is a reduced representation of the world.

A conceptual model is a set of well-defined concepts and their
interrelationships to obtain an abstracted representation of a
real-world entity.

We have to consider two important assumptions when designing a model for
a fault-tolerant real-time system:

\begin{itemize}
\item
  Load hypothesis: load offered to a computer system is below a maximum
  or peak load
\item
  Fault hypothesis: a statement about the assumptions that relate to the
  type and frequency of faults the computer system is supposed to
  handle.
\end{itemize}

Statements on the response time of a computer system can only be made
under the load hypotheses.

When designing a fault-tolerant system, only conditions described by the
fault hypothesis will guarantee a fault-tolerant behavior. Outside this
hypothesis, the system can fail.

\textbf{Notion of Physical Time:}

To a real-time system model, the notion of physical time is of central
importance. In our model, we assume that there is a omniscient external
observer with a precise reference clock z, and that all real-time clocks
in the nodes are synchronized within a precision .

\textbf{Duration of Actions:}

The execution of a statement constitutes an action.

We consider the following temporal quantities to describe temporal
behavior of an action a:

\begin{enumerate}
\def\labelenumi{\arabic{enumi}.}
\item
  Actual duration (actual execution time) : the number of time units of
  reference clock z that occur between start of action with input
  parameters x, and the termination of action a.
\item
  Minimal duration : smallest time to execute action \emph{a}, for all
  possible input parameters \emph{x}.
\item
  Worst-case execution time (WCET) : the maximum duration of action a
  for all conceivable input parameters.
\item
  Jitter: the difference between worst-case execution time and minimal
  execution time .
\end{enumerate}

\textbf{Frequency of Activations:}

The maximum number of activations of an action per unit of time we call
the frequency of activations.

Every processor (or computational resource) has a finite capacity. A
processor can meet its temporal obligations only if the frequency and
temporal distribution of the activations are controlled.

\textbf{What is not Part of the Model:}

\begin{itemize}
\item
  Representation of values (e.g. 4-20 mA, data format, etc.)
  Representation differences can be hidden behind gateways.
\item
  Details of data transformation (e.g. control algorithms,
  representation transformations, internal program logic).
\end{itemize}

\textbf{4.2} \protect\hypertarget{teil3}{}{}\textbf{The Structural
Elements }

A distributed fault-tolerant real-time system can be decomposed into a
set of communication clusters. A computational cluster can be
partitioned into a set of fault-tolerant units (FTU), connected by a
real-time communication network. Each FTU consists of one or more node
computers. Within a node computer, a set of concurrently executing tasks
run jobs, i.e. perform the intended functions (see figure on board).

\textbf{Jobs and Tasks}

A task is a sequential program. The execution of one or more tasks to
perform some function we call a job.

The time interval between the start of a job and its termination, given
an input data set x, is called the actual duration of the job on a given
target machine.

A task that does not have an internal state at its point of invocation
is called a stateless task.

Otherwise it is called a task with state.

\textbf{Simple Task (S-task):} a task with no synchronisation point
within, e.g. no semaphore ``wait'' operation, mutexes or other
constructs that could make the task wait or block. Execution time can be
determined in isolation, i.e. without knowledge of behaviour of other
tasks.

\textbf{Complex Task (C-task):} a task with blocking synchronization
statements, e.g. semaphore ``wait'' operation. Execution time is a
global issue, it can depend on progress of other tasks within the node.

\textbf{Node}

A node is a self-contained computer with its own hardware and software.

A node accepts input messages and produces the intended and timely
output messages via the communication network interface (CNI).

\includegraphics[width=4.40278in,height=2.58958in]{media/image1.png}

Data structures on the node can be categorized into initialization state
(i-state) and history state (h-state).

The i-state is a static data structure comprising the program code and
initialization data and can be stored in ROM.

The h-state is the dynamic data structure and is stored in RAM.

\textbf{Fault-Tolerant Unit (FTU)}

A fault tolerant unit consists of a set of replicated nodes that are
intended to produce replica determinate result messages.

In case one of the nodes of an FTU produces erroneous results, a
judgement mechanism detects this situation, and ensures that only
correct results are delivered to the clients of the FTU (e.g. voter
which takes majority of results).

From a logical point of view, an FTU can be considered as a single node.

\textbf{Computational Cluster}

A computational cluster comprises a set of FTUs that cooperate to
perform the intended fault-tolerant service.

Interfaces between a cluster and its environment (controlled object,
operator, other computational clusters) are formed by the gateway nodes
of the cluster.

Clusters can be interconnected by gateway nodes in form of a mesh
network.

\textbf{4.3} \protect\hypertarget{teil4}{}{}\textbf{Interfaces }

System architecture design is primarily interface design.

An interface between two subsystems of a real-time system is
characterized by

\begin{itemize}
\item
  The control properties (control signals crossing interface, what do
  they do)
\item
  The temporal properties (temporal constraints to be satisfied by
  control signals and data crossing the interface)
\item
  The functional intent (intended functions of the interface partner)
\item
  The data properties (structure and semantics of data elements crossing
  the interface)
\end{itemize}

In case events can occur any time, we speak of a dense time base.

In case events are permitted to occur only in certain time intervals π,
we speak of a sparse time base.

In a dense time base it can be difficult to establish temporal order.

Example: The functional intent of a node in an engine controller is to
guarantee the car conforms to environmental standards. As the standards
change, e.g. new laws are passed, the function of the node may have to
change, but not its functional intent.

The functional intent is thus at a higher level of abstraction than the
function.

The functional intent can be related to a ``goal'' in requirements
engineering.

\textbf{Resource Controllers}

In many cases, the interfacing partners use different syntactic
structures and incompatible coding schemes to represent information that
must cross the interface.

An intelligent interface component must be placed between the interface
partners to transform the different representations. This we call a
resource controller.

\includegraphics[width=8.17639in,height=1.13472in]{media/image2.png}

Resource controllers act as gateways between two different subsystems
with different representations.

\textbf{World and Message Interfaces}

We distinguish between real world interfaces, and message interfaces.

Example: specific man-machine interface (SMMI) would be a concrete world
interface, with touchpad, screen, buttons. A generalized man-machine
interface (GMMI or abstract message interface) would just consider the
messages that cross the interface, and their temporal properties.

\includegraphics[width=5.39653in,height=3.56042in]{media/image3.png}

\textbf{Standardized Message Interfaces}

To improve compatibility between systems designed by different
manufacturers, some international standard organizations have attempted
to standardize message interfaces. In automotive systems, the interfaces
are typically bus interfaces.

\textbf{Automotive bus systems (OSI level 0 to 2)}

\begin{longtable}[c]{@{}llll@{}}
\toprule
\textbf{Type of bus} & \textbf{Application} & \textbf{European
Standards} & \textbf{US Standards}\tabularnewline
\emph{Character based (UART)}\tabularnewline
K/L-Line & Diagnostics & ISO 9141 &\tabularnewline
SAE J1708 & Diagnostics, Class A On Board & & SAE J1708 (truck and bus,
9.6 kBit/s)\tabularnewline
LIN & Class A On Board & Manufacturer syndicate, 20
kBit/s\tabularnewline
\emph{PWM based}\tabularnewline
SAE J1850 & Diagnostics, Class A/B On Board & & SAE J1850 (PWM Ford,
VPWM GM), 10.4 and 41.6 kBit/s\tabularnewline
\emph{Bit stream based}\tabularnewline
CAN & Class B/C On Board & ISO 11898 1-3

Bosch CAN 2.0 A,B

47.6 \ldots{} 500 kBit/s

ISO 11992

CAN for trucks and trailers

ISO 11783 ISOBUS

CAN for agricultural vehicles & SAE J2284

(passenger cars)

500 kBit/s

SAE J1939

(truck and bus)

250 kBit/s\tabularnewline
TTCAN & Class C(+) On Board & ISO 11898-4

Bosch, 1 MBit/s &\tabularnewline
FlexRay & Class C+ On Board & Manufacturer syndicate, 10
MBit/s\tabularnewline
TTP & Class C+ On Board & Manufacturer syndicate\tabularnewline
MOST & Multimedia & Manufacturer syndicate, 25 MBit/s\tabularnewline
\bottomrule
\end{longtable}

\textbf{Automotive transport protocols (OSI level 4)}

\begin{longtable}[c]{@{}llll@{}}
\toprule
\textbf{Transport Protocol} & \textbf{Application} & \textbf{European
Standards} & \textbf{US Standards}\tabularnewline
ISO TP & CAN busses & ISO 15765-2 &\tabularnewline
SAE J1939 & CAN busses & & SAE J1939/21\tabularnewline
TP 1.6

TP 2.0 & CAN busses & Manufacturer standard

VW/Audi/Seat/Skoda

Base is OSEK COM 1.0 &\tabularnewline
\bottomrule
\end{longtable}

\textbf{Automotive application protocols (OSI level 7)}

\begin{longtable}[c]{@{}llll@{}}
\toprule
\textbf{Protocol} & \textbf{Application} & \textbf{European Standards} &
\textbf{US Standards}\tabularnewline
ISO 9141-CARB & Diagnostics

US OBD & ISO 9141-2

Outdated US diagnostics interface &\tabularnewline
KWP 2000

Keyword protocol & Diagnostics

(general and OBD) & ISO 14230

KWP 2000 diagnostics on K-Line

ISO 15765

KWP 2000 Diagnostics on CAN &\tabularnewline
UDS

Unified Diagnostic Services & Diagnostics

(general and OBD) & ISO 14229

UDS Unified Diagnostic Services &\tabularnewline
OBD & Diagnostics

US OBD

European OBD & ISO 15031

Identical to US standards & SAE J1930, J1962, J1978, J1979, J2012,
J2186\tabularnewline
CCP and XCP

CAN and Extended Calibration Protocol & Application & ASAM MCD 1

ASAM association &\tabularnewline
\bottomrule
\end{longtable}

\textbf{Temporal Obligations of Clients and Servers}

In the client-server model, a request (message) from a client to a
server causes a response from the server at a later time. Three temporal
parameters characterize such a client-server interaction:

\begin{enumerate}
\def\labelenumi{\arabic{enumi}.}
\item
  The maximum response time, RESP, that is expected by the client
\item
  The worst-case execution time, WCET, of the server
\item
  The minimum time, MINT, between two successive requests by the client
\end{enumerate}

The WCET is in the sphere of control of the server, and the MINT is in
the sphere of control of the client.

In a hard real-time environment the implementation must guarantee

WCET \textless{} RESP

as long as the client respects the specified MINT (e.g. no interrupt
overload).

\includegraphics[width=6.37431in,height=1.48958in]{media/image4.png}

\textbf{4.4} \protect\hypertarget{teil5}{}{}\textbf{Temporal Control
versus Logical Control}

Example rolling mill with three drive controllers and associated
controller nodes and pressure sensors:

\includegraphics[width=7.14792in,height=3.70625in]{media/image5.png}

\textbf{when} ((p1 \textless{} p2) \^{} (p2 \textless{} p3))

\textbf{then} everything ok

\textbf{else} raise pressure alarm;

But\ldots{}

\begin{enumerate}
\def\labelenumi{\arabic{enumi}.}
\item
  What is maximum tolerable time difference between occurrence of the
  alarm condition in the controlled object and the triggering of the
  alarm at the MMI?
\item
  What are the maximum tolerable time differences between the three
  pressure measurements in the different control nodes?
\item
  When do we have to activate the pressure measurement tasks at the
  drive controller nodes?
\item
  When do we have to activate the alarm monitoring task at the MMI node?
\end{enumerate}

The when statement intermingles two aspects: the point in time when the
alarm must be raised, and the logical condition that needs to be met to
raise the alarm.

Logical control is concerned with the control flow within a task, in
order to achieve the desired data transformation.

Temporal control is concerned with determining the points in time when a
task must be activated, or when a task must be blocked, because some
conditions outside the task are not satisfied at a particular moment.

For an S-task, the only temporal control issue is when to activate the
task.

A C-task combines issues of logical control and temporal control, for
example with a semaphore ``wait'' statement to delay program execution
until a temporal condition outside the task is satisfied.

Different approach: Synchronous languages like LUSTRE, ESTEREL. Task
finishes computation immediately, output is synchronous with input.

\textbf{Event-triggered versus Time-triggered}

A system where all control signals are derived from event triggers is
called an event-triggered (ET) system. Event examples: button pushed,
activation of limit switch, arrival of new message at a node, completion
of task within node.

A system where all control signals are derived from time triggers is
called a time-triggered (TT) system. A time trigger is a control signal
derived from the progression of time, e.g. the clock within a node
reaches a preset point in time.

Example: computer system controlling an elevator.

Many real-time systems use time triggers as well as event triggers, but
most favour one or the other.

\begin{itemize}
\item
  Advantage event-triggered systems: flexibility
\item
  Advantage time-triggered systems: predictability
\end{itemize}

\textbf{Interrupts}

In an ET system, the external event triggers are often relayed to the
computer system via interrupts. An interrupt is an asynchronous
hardware-supported request for a specific task activation caused by an
event external (i.e. outside the node) to the currently active
computation.

\includegraphics[width=6.64583in,height=2.64583in]{media/image6.png}

Interrupts cause an overhead (worst case administrative overhead, WCAO),
mostly due to the required context switches.

If interrupt frequency is too high, no CPU capacity remains for actual
computations. Interrupts are outside the sphere of control of the node,
which makes it difficult to handle such overload conditions.

\textbf{Trigger Task}

In a TT system, to observe state changes outside the computer the state
of the environment is regularly captured by a trigger task.

The result of a trigger task can be a control signal that activates
other tasks.

Only states with a duration greater than the sampling period of the
trigger task can be observed reliably. Short lived states (push of
button) have to be stored in a memory element within the interface.

The periodic trigger task generates an administrative overhead in a TT
system, regardless of any events that may occur. The administrative
overhead becomes large if the sampling period of the trigger task gets
small (\textless{} 1ms). The period of a trigger task must be smaller
than the laxity (difference between deadline and execution time) of an
RT transaction caused by an event in the environment.

(Illustration).

\textbf{4.5} \protect\hypertarget{teil6}{}{}\textbf{Worst Case Execution
Time}

Deadlines can only be guaranteed if worst case execution (WCET) times of
all application tasks are known a priori.

In addition, the worst case delays caused by administrative functions
(e.g. operating system services, context switches, scheduling) need to
be known. These delays we call the worst case administrative overhead
(WCAO).

\textbf{WCET of non-preemptive S-Tasks}

WCET depends on

\begin{itemize}
\item
  source code of task
\item
  properties of object code generated by compiler
\item
  characteristics of compiler
\end{itemize}

\textbf{Source code analysis:}

Problem is to find the longest path through the code (critical path).
Number of paths increases exponentially with size of program.
Annotations provided by the programmer can assist a tool to extract the
critical path, by providing additional semantic information.

\textbf{Compiler analysis:}

The execution time of source code statements depends heavily on the
object code generated by the compiler. The compiler generated object
code timing analysis must be related to the source program by means of
statement-level annotations.

\textbf{Microarchitecture timing analysis:}

The execution time of the object code generated by the compiler depends
on the microarchitecture of computer that executes it.

For small, simple microprocessors it is possible to predict the time
required to execute a single machine language construct by looking it up
in the specifications.

For larger microprocessors, with their pipelining and caching mechanism,
it is very difficult to predict reliably an upper bound on the execution
times of object code. Dependencies among instructions can cause pipeline
hazards and cache misses.

The best that has been achieved to date is WCET bound 100\% above the
measured value.

\textbf{WCET of preemptive S-Tasks}

The execution time of an S-task is extended by three terms in case of
preemption:

\begin{itemize}
\item
  The WCET of the interrupting task
\item
  The WCET of the operating system for context switching
\item
  The time required for reloading the instruction and data cache after
  the context switches
\end{itemize}

The sums of the last two terms we call the worst case administrative
overhead (WCAO).

For modern microprocessors, the last component contributes significantly
to the WCAO.

\includegraphics[width=6.15000in,height=2.87917in]{media/image7.png}

\textbf{WCET of C-Tasks}

The WCET of C-tasks does not only depend on performance of the task
itself, but also on thebehaviour of other tasks and the operating system
within a node.

WCET analysis of a C-task is thus a global problem involving all
interacting tasks within a node. In addition to preemption, intended
task dependencies (mutual exclusion, precedence) have to be considered.

\textbf{State of Practice}

In practice, it is usually not feasible to obtain a tight upper bound on
the WCET for any but the most simple system. However, since bounds for
WCETs are required in a hard real-time application, the problem is
solved by combining some of the following techniques:

\begin{enumerate}
\def\labelenumi{\arabic{enumi}.}
\item
  Gather experimental data for WCET analysis by measuring a similar
  actual implementation.
\item
  Restrict architecture to reduce interactions among tasks and to
  facilitate the a priori analysis of the control structure. Minimize
  explicit synchronizations that require context switches.
\item
  Analyze subproblems (e.g. maximum execution time analysis of the
  source program) to obtain a set of test cases biased towards the worst
  case execution time.
\item
  Do extensive testing of the complete implementation to measure the
  safety margin bet­ween the assumed WCET and actual measured execution
  times.
\end{enumerate}

In practice, since it is close to impossible ascertain that an assumed
upper bound on the WCET is the actual WCET, one relies on estimates
based on experience and extensive testing.

\textbf{4.6} \protect\hypertarget{teil7}{}{}\textbf{The History State}

The h-state of any point of interruption can be defined as the contents
of the program counter and of all data structures that must be loaded
into a ``virgin'' hardware device to resume operation at the point of
interruption.

\includegraphics[width=4.90069in,height=2.51181in]{media/image8.png}

\includegraphics[width=5.64931in,height=1.74167in]{media/image9.png}

Example: pocket calculator calculation a sine(x). H-state is initially
empty, and when result is displayed, is empty again. In between the
series expansion of the sine takes place. If the computational device
was halted anywhere between start and end, we could observe that the
h-state is not empty.

Size of the h-state depends on the level of abstraction, and the time of
observation. The size of the h-state can be reduced if granularity of
observations is increased, and if observation points are selected
immediately before or after an atomic operation.

A small h-state at the reintegration point simplifies reintegration of a
failed component.

\textbf{Ground state}

The ground state is a state where no tasks are active and all
communication channels are flushed. Reintegration of a node after
failure is simplified if a node periodically visits a ground state that
can be used as a reintegration point.

\includegraphics[width=4.89653in,height=2.30139in]{media/image10.png}

\protect\hypertarget{teil8}{}{}\textbf{Points to Remember}
