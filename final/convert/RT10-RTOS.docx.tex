\begin{enumerate}
\item ~
  \section{Chapter 10}\label{chapter-10}

  \begin{enumerate}
  \item ~
    \subsection{Real-Time Operating
    Systems}\label{real-time-operating-systems}
  \end{enumerate}
\end{enumerate}

Overview 1

\begin{enumerate}
\def\labelenumi{\arabic{enumi}.}
\item
  Task Management 3
\item
  Interprocess Communication 12
\item
  Time Management 16
\item
  Error Detection 21
\item
  Case Study: OSEK and AUTOSAR 29
\end{enumerate}

Points to Remember 32

\begin{enumerate}
\item ~
  \subsection{}\label{section}

  \subsection{Overview}\label{overview}
\end{enumerate}

\begin{itemize}
\item
  Introduction to TTP/C
\item
  Protocol objectives
\item
  Protocol layers
\item
  Smallest replaceable units and fault tolerant units
\item
  Structure of the communication node interface (CNI)
\item
  Membership service and CRC computation
\item
  Introduction to TTP/A for field bus applications
\end{itemize}

\textbf{10.1} \protect\hypertarget{teil2}{}{}\textbf{Task Management}

Time-Triggered Protocols (TTP) are designed for hard real-time systems.

There are two versions:

\begin{itemize}
\item
  TTP/C for fault-tolerant hard real-time systems
\item
  TTP/A for low cost applications (e.g. field bus applications)
\end{itemize}

Example application: Pinoth snow groomer

\includegraphics[width=6.59653in,height=3.75625in]{media/image1.png}

Example application: Boeing 787 Dreamliner

\includegraphics[width=7.63611in,height=5.49583in]{media/image2.png}

\textbf{Protocol Objectives}

\begin{enumerate}
\def\labelenumi{\arabic{enumi}.}
\item
  Message transport with low latency and minimal jitter
\item
  Support of Composability
\item
  Provision of a fault-tolerant membership service
\item
  Fault-tolerant clock synchronization
\item
  Distributed redundancy management
\item
  Minimal overhead, both in message length and in number of messages
\item
  Scalability to high data rates, efficient operation on twisted-pair
  and optical fiber
\end{enumerate}

\textbf{Structure of a TTP System}

A TTP system consists of a cluster of fault-tolerant units (FTU's), each
comprised of one or more nodes, and interconnected by a communication
network.

\includegraphics[width=8.76319in,height=2.78819in]{media/image3.png}

A node is the smallest replaceable unit (SRU).

A node consists of two subsystems:

\begin{itemize}
\item
  the host computer
\item
  the communication controller
\end{itemize}

\includegraphics[width=7.39236in,height=3.39167in]{media/image4.png}

The communication network interface (CNI) is the node-internal interface
between the communication controller and the host. The CNI is formed by
a dual-ported RAM.

Data integrity passed between host and communication controller is
ensured by a non-blocking write protocol.

The communication controller has a local memory to hold the message
descriptor list (MEDL).

The MEDL determines

\begin{itemize}
\item
  at what point in time a node is allowed to send a message
\item
  when a node can expect to receive a message
\end{itemize}

The MEDL has the size of one cluster cycle.

\includegraphics[width=9.90556in,height=2.23125in]{media/image5.png}

The Bus Guardians (BGs) are independent hardware devices. They monitor
temporal access pattern of the controller to the replicated busses, and
terminate controller operation in case of timing violations.

\textbf{Design Rationale}

TTP is a TDMA protocol. Every node sends a message on a shared
communication channel during a predetermined, statically assigned time
slot.

\textbf{Composability:}

The TTP controller operates autonomously. It is controlled by the MEDL
inside the controller, and the fault-tolerant global time.

The CNI between the TTP controller and the host computer is fully
specified in the value and temporal domain, which supports composability
of an architecture.

An error in any one of the hosts cannot interfere with proper operation
of the communication system, since no control signals cross the CNI. The
MEDLs are inaccessible to the hosts.

\textbf{Best use of a priori knowledge:}

In a time-triggered architecture it is known at design time which node
must send which message at what time. Thus, a receiver can detect
missing messages, or bus guardians can prevent the effect of babbling
clients.

\textbf{Naming:}

In a time-triggered architecture, messages can be uniquely identified by
their time of transmission. Thus there is no need to transport message
names along with the message. This increases data efficiency,
particularly with short messages. It also improves composability, since
there is no need to agree on message names across different hosts.

\textbf{Acknowledgement scheme:}

Every operational member of an ensemble hears every message transmitted
on the communication channel. As soon as one receive acknowledges the
message, it can be concluded that the message has been sent correctly,
and that all correct receivers will have received the mes­sage.

\textbf{Fail silence in the temporal domain:}

TTP assumes that nodes support the file-silent abstraction in the
temporal domain. This means a node either delivers a message at the
correct time, or not at all.

This supports error confinement at the system level. The bus guardians
implement the fail-silent mode for nodes by monitoring channel access
and disconnection the channel from the controller in case of temporally
erroneous access.

A membership service can detect the failure of a node with small
latency.

\textbf{Fail silence in the value domain:}

Fail silence in the value domain must be implemented by the host. For
this, the host must ensure by space and/or time redundancy that all
internal failures of a host are detected before a non-detectable
erroneous output message is transmitted.

Value failures at the communication level are detected by a CRC
mechanism.

\textbf{Design tradeoffs:}

TTP has decided to limit bandwidth requirements on cost of processing
requirement at the nodes. The rationale behind this decision is that
bandwidth is more difficult to increase than processing power, which has
steadily increased over the past years with advancements in VLSI
technology.

\textbf{Protocol Variants}

TTP comes in two variants, a full version called TTP/C and a scaled-down
version called TTP/A. At the CNI, the structure is compatible for both
protocol versions.

\textbf{TTP/C:}

TTP/C provides all services needed for the implementation of a
fault-tolerant distributed real-time system. TTP/C supports FTUs that
comprise replicated communication channels and different replication
strategies, e.g., replicated fail-silent nodes or TMR nodes.

TTP/C requires a specially designed communication controller containing
hardware mechanisms for the implementation of the protocol functions.

\textbf{TTP/A:}

TTP/A is intended for non fault-tolerant field bus applications. It
requires just a standard UART hardware port and a local real-time clock,
typically available on most low-cost microcontrollers. The protocol
logic is implemented in the software of a microcontroller.

The following table compares features for the two protocol variants.

\begin{longtable}[c]{@{}lll@{}}
\toprule
Service & TTP/A & TTP/C\tabularnewline
Clock synchronization & Central multimaster & Distributed,
fault-tolerant\tabularnewline
Mode switches & Yes & Yes\tabularnewline
Communication error detection & Parity & 16/24 bit CRC\tabularnewline
Membership service & simple & full\tabularnewline
External clock synchronization & yes & yes\tabularnewline
Time-redundant transmission & yes & yes\tabularnewline
Duplex nodes & no & yes\tabularnewline
Duplex channels & no & yes\tabularnewline
Redundancy management & no & yes\tabularnewline
Shadow node & no & yes\tabularnewline
\bottomrule
\end{longtable}

\textbf{10.2} \protect\hypertarget{teil3}{}{}\textbf{Interprocess
Communication }

The TTP/C is organized in a set of conceptual protocol layers, which do
not correspond to the OSI protocol layers. The Basic Communication
Network Interface (CNI) typically marks the interface between purely
software implemented protocol functionality and software/hardware
related functionality.

\includegraphics[width=5.18264in,height=4.50278in]{media/image6.png}

\textbf{Data Link/Physical Layer}

This layer serves to exchange frames between nodes. It provides

\begin{itemize}
\item
  media access control
\item
  bit synchronization
\item
  bit encoding and bit decoding
\end{itemize}

Access scheme is time-division-multiple access (TDMA). Access is
controlled by data stored in the MEDL of the TTP controller.

Bit synchronization and bit encoding/decoding uses Modified Frequency
Modulation (MFM) code. In MFM, data bits are separated by clock bits;
(x,y) encodes to (x, x NOR y, y).

Requirements on physical layer on twisted pair are less demanding than
those of a CAN system (no bit arbitration required). Thus, a CAN twisted
pair network can be used.

Today, data rates specified go up to 25 MBit/s.

\textbf{SRU Layer}

The single replaceable unit (SRU) layer provides the following services:

\begin{itemize}
\item
  Stores data fields of received frames in the memory area of the CNI
  DPRAM
\item
  Establishes node membership
\item
  Performs Byzantine-resilient clock synchronization by the
  fault-tolerant average algorithm
\item
  Provides immediate and deferred mode change service to higher layers
\end{itemize}

\textbf{Redundancy Management Layer (RM Layer)}

The redundancy management layer provides the following services:

\begin{itemize}
\item
  Cold start of a TTP/C cluster. For this it uses the mode-change
  service of the SRU layer.
\item
  Reintegration of a repaired node
\item
  Dynamic redundancy management, i.e. replacement of a failed node by a
  shadow node
\end{itemize}

The CNI contains a node reconfiguration field. In case a host decides to
assume a new role, the name of the requested role is written into the
reconfiguration field. The TTP controller checks if that role is
permitted. If that is the case, it performs a node role change, and
reconfigures the bus guardians to protect bus access in the new role.

\includegraphics[width=8.51736in,height=2.71528in]{media/image7.png}

\textbf{FTU Layer}

The FTU layer groups two or more nodes to form a fault-tolerant unit.

The FTU layer ensures that data are only visible in the FTU CNI after
they have become permanent.

Examples for different FTU layer implementations:

\begin{enumerate}
\def\labelenumi{\arabic{enumi}.}
\item
  Two fail-silent nodes can be grouped into an FTU that provides the
  specified service as long as one of the two nodes is operational.
  Fail-silence in the value domain is ensured by the host.
\item
  Three nodes can be grouped into a Triple Mode Redundancy (TMR) FTU. A
  TMR FTU can tolerate a single value failure in any of its nodes.
  Synchronization of the three nodes is implemented by the lower layers.
\item
  A FTU can comprise of software subsystems executing on different
  nodes.
\end{enumerate}

The FTU membership service is provided by the FTU layer.

The FTU layer can be implemented in the host computer or in the TTP/C
controller.

A basic TTP/C controller, implemented in hardware, does not include the
FTU layer, but rather provides the Basic CNI interface to the software
in the host computer.

\textbf{10.3} \protect\hypertarget{teil4}{}{}\textbf{Time Management }

The CNI is the only interface of the communication system that is
visible to the software of the host computer. It is the programming
interface of the TTP network.

The CNIs for the TTP/C and TTP/A protocols are upward compatible.

\includegraphics[width=6.81667in,height=2.21042in]{media/image8.png}

\textbf{Structure of the CNI}

The CNI is a data-sharing interface between the RM layer and the FTU
layer. It consists of

\begin{itemize}
\item
  a status/control area, containing system information. It serves to
  facilitate communication between the host computer and the TTP
  controller via dedicated data fields
\item
  a message area, containing the messages sent or received by the node,
  including a control byte for each message
\item
  a control line from the TTP controller to the host computer,
  signalling the tick of the global clock
\end{itemize}

Optionally, there may be another control line from the TTP controller to
the host to signal that a certain time has been reached.

\textbf{Status/Control Area}

The status/control area of the CNI is a memory area of the DPRAM
containing the control and status information shared between the TTP
controller and the host computer.

\textbf{Status registers updated by the TTP controller:}

The following status registers are updated by the TTP controller:

\begin{itemize}
\item
  Global time (two byte) of the cluster, established by the mutual
  internal synchronization of the TTP controllers.
\item
  SRU-time, containing the current global time in SRU slot granularity.
  This time stays constant during a complete SRU slot and increases at
  the beginning of the next SRU slot.
\item
  MEDL position, denotes the current operating mode of the cluster and
  the current position in the message descriptor list.
\item
  Node membership vector, comprises as many bits as there are nodes in
  the cluster. Each node is assigned to a specified bit position of the
  membership vector. When the bit is set to ``TRUE'', the node was
  operational during its last sending slot. The membership is adjusted
  at the end of each SRU slot after all messages from the sending node
  must have arrived and the CRC fields have been analyzed.
\item
  Status information, diverse status and diagnosis information regarding
  operation of the protocol
\end{itemize}

The SRU-time, MEDL position, and node membership vector form the h-state
of the communication controller (called C-state). The protocol only
operates correctly if all members of the ensemble have the same C-state.

The protocol continually enforces synchronization of the C-state between
sender and receiver.

\textbf{Control registers written by the host:}

The following control registers are written by the host CPU:

\begin{itemize}
\item
  Watchdog field, must be periodically updated by the host CPU. The TTP
  controller checks this field to determine if the host CPU is alive. If
  the CPU does not update this field anymore, the controller assumes
  that the CPU has a failure and stops sending messages on the network.
\item
  Time-out register, permits the host computer to request a temporal
  control signal (an interrupt) at a specific future point of the global
  time. This permits the host to synchronize activities with the global
  time in the cluster.
\item
  Mode-change request, can be used to request a mode change (e.g.,
  airplane on the ground or in the air) to a new schedule in all nodes
  of a cluster. This mode change request is transmitted to all other
  nodes at the next sending point of this node. TTP permits immediate
  mode changes and deferred mode changes. A deferred mode change is
  delayed until the start of the next cluster cycle. A mode change is
  potentially dangerous. The MEDL of the communication controller
  contains a static lock that can be turned on before system start so
  that mode changes are disabled.
\item
  Reconfiguration request, permits the host CPU to request a role change
  of the node. In case a host detects that an important node has failed
  the host can execute a role change request to perform the function of
  the failed node. To avoid erroneous role changes, the role-change
  mechanism is protected by special permission fields in the MEDL.
\item
  External rate correction is provided for external clock
  synchronization. This permits a time gateway to request a bounded
  common-mode drift of all nodes in a cluster to syn­chronize with an
  external time source, such as a GPS time receiver.
\end{itemize}

\textbf{Message Area}

The message area is application specific and determined by the MEDL of
the TTP controller. Beyond the actual message data, a message entry
contains a status byte indicating potential error conditions.

\includegraphics[width=6.45278in,height=0.46250in]{media/image9.png}

\textbf{Consistent Data Transfer}

Since access to the CNI can be concurrent, inconsistencies could result
when at the same time data is being written and read.

Consistency of a single-word data transfers across the CNI is guaranteed
by the hardware arbitration of the DPRAM. Multi-word data-transfer is
realized as follows:

\textbf{Controller to host:}

When a message arrives, it is copied from the receive buffer of the TTP
controller into the message area of the CNI. The time when this happens
is known a priori.

Together with the message data field, a status byte is written by the
TTP controller. All this occurs before the end of each SRU slot.

Access conflicts can be avoided in two ways:

\begin{enumerate}
\def\labelenumi{\arabic{enumi}.}
\item
  The host avoids reading data from the CNI during the times it knows
  the TTP controller writes data to the CNI. This is possible if the
  host can derive its read-access intervals from the global time base.
\item
  A non-blocking write protocol is being used. When the TTP controller
  writes to the common data area, a flag is set indicating that a write
  is in progress. When the host computer tries to read data from the
  shared area, and detects during any single read operation that the
  write flag has been set, it stops the read operation and retries with
  a small delay. This protocol ensures that the TTP controller is never
  delayed while accessing the CNI.
\end{enumerate}

\textbf{Host to controller:}

The host needs to be aware of the current time and knows a priori when
the TTP controller reads from the CNI. The host operating system must
synchronize its output action such that it does not write into the CNI
when the TTP controller performs a read operation.

\textbf{10.4} \protect\hypertarget{teil5}{}{}\textbf{Error Detection}

\textbf{The Message Descriptor List (MEDL)}

The Message Descriptor List (MEDL) is a static data structure within
each TTP controller that

\begin{itemize}
\item
  controls when a message must be sent on or received from the
  communication channel
\item
  contains the position of the data in the CNI
\end{itemize}

\includegraphics[width=4.76042in,height=1.49306in]{media/image10.png}

The MEDL serves as a dispatching table for the TTP controller. The
length of the MEDL is determined by the length of the cluster cycle,
i.e., the sequence of TDMA rounds after which the operation of the
cluster repeats itself.

\textbf{MEDL entry:}

An entry in the MEDL comprises three fields:

\begin{itemize}
\item
  a time field (SRU-time), with the point in global time with SRU
  granularity when the message specified in the address field must be
  communicated
\item
  an address field, points to the CNI memory cells where the data items
  must be stored to or retrieved from
\item
  an attribute field with four subfields:
\item
  \begin{quote}
  a direction subfield (D), specifying if the message is an input or
  output message
  \end{quote}
\item
  \begin{quote}
  a length subfield (L), denoting the length of the message that must be
  communicated
  \end{quote}
\item
  \begin{quote}
  an initialization subfield (I) specifying whether the message is an
  initialization message or a normal message
  \end{quote}
\item
  \begin{quote}
  a parameter subfield (A) containing additional protective information
  concerning mode changes and node role changes
  \end{quote}
\end{itemize}

The host can only execute mode-changes permitted by the attribute field
of the MEDL. In safety-critical systems, all mode changes requested by a
host can be blocked by the MEDL.

Each node must have its personal MEDL. The set of all personal MEDLs of
a cluster must be consistent.

MEDLs are generated during design time by cluster compilers.

\textbf{Name mapping:}

Name mapping is performed under the control of the MEDL in each
controller. Unambiguous identification of each message is given by its
global transmission time. Thus, name mapping from the sending host to
the receiving host is performed

\begin{itemize}
\item
  from host to CNI memory cell at sender and unique global transmission
  time
\item
  from CNI memory cell to host at receiver and unique global
  transmission time
\end{itemize}

\includegraphics[width=7.26111in,height=2.88542in]{media/image11.png}

\textbf{Frame Format}

A TTP/C frame consists of three fields:

\begin{itemize}
\item
  a four-bit header
\item
  a variable length data field of up to sixteen bytes (length is defined
  in MEDL)
\item
  a two or three byte CRC field
\end{itemize}

\includegraphics[width=7.62917in,height=1.55000in]{media/image12.png}

During normal operation, a node transmits one such frame during an SRU
slot on each of the two replicated channels.

\textbf{First bit of the header:}

This bit determines if the message is a normal (N) message or an
initialization (I) message. I-messages are used to initialize the
system. They carry the C-state of the sender in the data field and make
it possible for a new node to get the current C-state of the protocol
when joining the ensemble.

\textbf{Mode bits:}

These bits can be used to request a mode change in all nodes of the
cluster. The change is differential; one out of seven successor modes to
any given mode can be selected. The mode change mechanism can be
disabled by setting parameters in the MEDL.

\textbf{Data field:}

This field contains up to sixteen data bytes. The actual length is given
by an entry in the MEDL.

\textbf{CRC field:}

This field contains the CRC check sum for communication error detection.

\textbf{CRC Calculation}

The CRC of an I-message is calculated of the concatenation of the header
and the data bytes.

For N-messages the procedure is slightly different.

At the sender, the CRC is calculated over the message contents
concatenated with the sender's C-state.

At the receiver, the CRC is calculated over the received message
contents concatenated with the receivers C-state.

If the result of the CRC at the receiver is negative, then either the
message has been corrupted during transmission, or there is a
disagreement between the C-states of the sender and the receiver. In
both cases, the message must be discarded.

\includegraphics[width=6.50417in,height=2.99097in]{media/image13.png}

\textbf{The Membership Service}

The membership field of a node contains one bit for each cluster node.
If one out of the redundant messages is received by a receiving node,
the receiving node considers the sender node operational at this
membership point.

The node is considered operational until its next membership point in
the following TDMA cycle. If a node fails within this interval, the
failure is only recognized at the next membership point. Thus, the delay
of the membership information is at most one TDMA cycle.

If none of the expected messages arrives with a correct CRC, then a
receiver considers the sending node as failed and clears the membership
bit of the sender at the end of the current SRU slot.

If a particular node did not receive any correct message from a sending
node, e.g., because the incoming link of the receiver has failed, it
assumes that this sending node has crashed, and it eliminates it from
its membership vector at the end of the current SRU slot.

If, however, all other nodes received at least one these messages they
come to a different conclusion about the membership. This creates two
cliques that cannot communicate with each other because they contain a
different C-state (the membership vector is part of the C-state).

TTP contains a mechanism which let the majority view in such a conflict
situation, i.e., the node with the failed input port, which is in the
majority, is removed from the membership. Before sending a message, a
node counts its negative CRC results during the last TDMA round. If more
than half of the messages received have been discarded because of a
failed CRC, the node assumes that its C-state differs from the majority,
terminates its operation and thus leaves membership.

This mechanism avoids clique formation among the nodes of the ensemble.
Agreement on membership is thus bound to indirect acknowledgement of
message reception by the majority.

\textbf{Clock Synchronization}

Clock synchronization within TTP does not require any special
synchronization messages or send time information as part of a message.

Each node knows from its MEDL the expected time of arrival of each
message. The deviation between the observed time of arrival to the
expected time of arrival is thus a measure for the clock difference
between the receiver's clock and the sender's clock.

A fault-tolerant clock synchronization algorithm is applied periodically
to synchronize the global time of each node in a cluster and keep it
within a predefined precision.

\textbf{10.5} \protect\hypertarget{teil6}{}{}\textbf{Case Study: OSEK
and AUTOSAR}

Deadlines can only be guaranteed if worst case execution (WCET) times of
all application tasks are known a priori.

In addition, the worst case delays caused by administrative functions
(e.g. operating system services, context switches, scheduling) need to
be known. These delays we call the worst case administrative overhead
(WCAO).

\textbf{Principles of Operation}

WCET depends on

\begin{itemize}
\item
  source code of task
\item
  properties of object code generated by compiler
\item
  characteristics of compiler
\end{itemize}

\textbf{Round:}

Problem is to find the longest path through the code (critical path).
Number of paths increases exponentially with size of program.
Annotations provided by the programmer can assist a tool to extract the
critical path, by providing additional semantic information.

\includegraphics[width=8.01736in,height=1.91528in]{media/image14.png}

\textbf{Modes:}

The execution time of source code statements depends heavily on the
object code generated by the compiler. The compiler generated object
code timing analysis must be related to the source program by means of
statement-level annotations.

\textbf{Time-outs:}

The execution time of source code statements depends heavily on the
object code generated by the compiler. The compiler generated object
code timing analysis must be related to the source program by means of
statement-level annotations.

\textbf{Error Detection and Error Handling}

WCET depends on

\textbf{Error detection in the time domain:}

The execution time of source code statements depends heavily on the
object code generated by the compiler. The compiler g

\textbf{Error detection in the value domain:}

The execution time of source code statements depends heavily on the
object code generated by the compiler. The compiler g

\textbf{Response Time of a TTP/A System}

WCET depends on

\protect\hypertarget{teil9}{}{}\textbf{Points to Remember}
