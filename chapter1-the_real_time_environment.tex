%%
%% Copyright (C) Hochschule Esslingen, 2011
%%
%% $Id: example.tex 131 2011-10-16 01:11:22Z uwe $
%%
%% This work may be distributed and/or modified under the
%% conditions of the LaTeX Project Public License, either version 1.3
%% of this license or (at your option) any later version.
%% The latest version of this license is in
%%   http://www.latex-project.org/lppl.txt
%% and version 1.3 or later is part of all distributions of LaTeX
%% version 2005/12/01 or later.


\begin{document}

% If you wish to uncover everything in a step-wise fashion, uncomment
% the following command: 
% \beamerdefaultoverlayspecification{<+->}

\mode<article>{\maketitle}


\begin{frame}
  \titlepage
\end{frame}

\begin{frame}{Inhaltsverzeichnis}
  \tableofcontents
\end{frame}


\begin{frame}{Summary} 
\textbf{Chapter 2: Distributed Real-Time Environment}
\begin{itemize}
	\item Distribute system architecture overview, clusters, nodes, communication
  network
 	\item Structure of node with host computer, communication network interface,
  communication controller
	\item	Event and state messages, gateways. 
	\item	Concept of composability. 
	\item	Event- and time-triggered communication systems. 
	\item	Scalability, dependability, issues of physical installation.
\end{itemize}
\end{frame}

\begin{frame}{Summary}
\textbf{Chapter 3: Global time}
\begin{itemize}
\item Notions of causal order, temporal order, and delivery order
\item External observers, reference clocks, and global time base
\item Sparse time base to view event order in a distributed real-time system
\item Internal clock synchronization to compensate for drift offset. Influence of the commu-nication system jitter on the precision of the global time base.
\item External time synchronization, time gateways, and the Internet network time protocol (NTP)
\end{itemize}
\end{frame}

\begin{frame}{Summary}
\textbf{Chapter 4: Modeling Real-Time Systems}
\begin{itemize}
\item Introduction of a conceptual model for real-time systems
\item Tasks, nodes, fault-tolerant units, clusters
\item Simple and complex tasks
\item Interface placement and interface layout
\item Temporal control and logical control
\item The history state
\end{itemize}
\end{frame}

\begin{frame}{Summary}
\textbf{Chapter 5: Real-Time Entities and Images}
\begin{itemize}
\item Real-time entities
\item Observations, state and event observations
\item Real-time images as current picture of real-time entity, and real-time objects
\item Temporal accuracy and state estimation to improve real-time image accuracy
\item Permanence in case of race conditions and idempotency with replicated messages
\item Replica determinism to implement fault-tolerance by active redundancy
\end{itemize}
\end{frame}

\begin{frame}{Summary}
\textbf{Chapter 6: Fault Tolerance}
\begin{itemize}
\item Failures, Errors, and Faults
\item Error Detection
\item A Node as a Unit of Failure
\item Fault Tolerant Units
\item Reintegration of a Repaired Node
\item Design Diversity
\end{itemize}
\end{frame}

\begin{frame}{Summary}
\textbf{Chapter 7: Real-Time Communication}
\begin{itemize}
\item Real-Time Communication Requirements
\item Flow Control
\item OSI Protocols for Real-Time
\item Fundamental Conflicts in Protocol Design
\item Media-Access Protocols
\item Performance Comparison: ET versus TT
\item The Physical Layer
\end{itemize}
\end{frame}

\begin{frame}{Summary}
\textbf{Chapter 8: Time-Triggered Protocols}
\begin{itemize}
\item Introduction to Time-Triggered Protocols
\item Overview of the TTP/C Protocol Layers
\item The Basic CNI
\item Internal Operation of TTP/C
\item TTP/A for Field Bus Applications
\end{itemize}
\end{frame}

\begin{frame}{Summary}
\textbf{Chapter 9: Input and Output}
\begin{itemize}
\item The dual role of time
\item Agreement protocol
\item Sampling and polling
\item Interrupts
\item Sensors and actuators
\item Physical installation
\end{itemize}
\end{frame}

\begin{frame}{Summary}
\textbf{Chapter 10: Real-Time Operating Systems: OSEK and AUTOSAR}
\begin{itemize}
\item Task management
\item Interprocess communication
\item Time management
\item Error detection
\item OSEK and AUTOSAR
\end{itemize}
\end{frame}

\begin{frame}{Summary}
\textbf{Chapter 11: Real-Time Scheduling}
\begin{itemize}
\item The scheduling problem
\item The adversary problem
\item Dynamic scheduling, dynamic priority servers
\item Static scheduling, fixed priority servers
\end{itemize}
\end{frame}

\begin{frame}{Summary}
\textbf{Chapter 12: Validation}
\begin{itemize}
\item Building a Convincing Safety Case
\item Formal Methods
\item Testing
\item Fault Injection
\item Dependability Analysis
\end{itemize}
\end{frame}


\section{Bibliography}

\begin{frame}
    \frametitle{Bibliography}
\tiny
	Text books for this lecture\\
\vspace*{1ex}

  \begin{thebibliography}{9}

  \beamertemplatearticlebibitems

    \bibitem{beameruserguide}
    Kopetz, H: Distributed Real-Time Systems, Springer, $2008$
    \bibitem{beameruserguide}
    Buttazo, G: Hard Real-Time Computing Systems, Springer $2005$
  \end{thebibliography}

\vspace*{2ex}
Complementing texts
Some books complementing the material treated in this lecture

\vspace*{1ex}

  \begin{thebibliography}{9}
  \beamertemplatearticlebibitems

    \bibitem{beameruserguide}
Liu, J.S.:	Real-Time Systems, Prentice Hall $2000$
    \bibitem{beameruserguide}
Verissimo, P; Rodrigues, L.: 	Distributed Systems for System Architects,
Kluwer $2001$
    \bibitem{beameruserguide}
Laplante, P.:	Real-Time Systems Design and Analysis, IEEE Press, $2004$
    \bibitem{beameruserguide}
Halbwachs, N.:	Synchronous Programming of Reactive Systems, Kluwer $1993$
    \bibitem{beameruserguide}
Zimmermann, W.; Schmidgall, R.:	Bussysteme in der Fahrzeugtechnik, Vieweg
$2006$ (German only)

  \end{thebibliography}

\vspace*{2ex}

Journal Articles and Web Documents
Original journal articles and documents from the web pertaining to this lecture

\vspace*{1ex}

  \begin{thebibliography}{9}
  \beamertemplatearticlebibitems

    \bibitem{beameruserguide}
    Albert, A.: Comparison of Event-Triggered and Time-Triggered Concepts with
    Regard to Distributed Control Systems, Embedded World, $2004$, Nuernberg,
%    \url{http://www.semiconductors.bosch.de/pdf/embedded_world_$04$_albert.pdf}
    \bibitem{beameruserguide}
    Mueller, B.; Fuehrer, T.; Hartwich, F.; Huegel, R.; Weiler, H.:    Fault
    Tolerant TTCAN Networks, Proceedings $8$th International CAN Conference; $2002$; Las
    Vegas, NV %http://www.semiconductors.bosch.de/pdf/Fault_Tolerant_TTCAN.pdf



  \end{thebibliography}

\end{frame}
\end{document}
